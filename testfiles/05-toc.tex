\input{regression-test.tex}
\PassOptionsToClass{fontset=fandol}{thuthesis}

\documentclass[degree=doctor]{thuthesis}

\begin{document}
\START
\showoutput

\frontmatter
\setcounter{page}{3}
\tableofcontents
\clearpage
\OMIT

\mainmatter
\chapter{引言}
\section{对等网络概述}
\subsection{P2P简单发展历史回顾}
\subsection{P2P研究的关键问题}
\section{对等网络中的搜索技术}
\subsection{P2P搜索算法的分类和当前进展情况}
\subsection{研究P2P中高效率、功能多样化的搜索算法之必要性}
\subsection{P2P中信息搜索的困难和挑战}
\section{本文研究的主要内容和主要贡献}
\subsection{研究什么和不研究什么}
\subsection{各章内容简介}
\subsection{本文的主要贡献}

\chapter{相关工作}
\section{对等网络基础设施}
\subsection{非结构化P2P系统与非收敛性路由}
\subsection{结构化P2P系统与收敛性路由}
\subsection{P2P系统中的数据存放策略}
\section{传统集中式环境中的信息搜索}
\subsection{信息索引技术}
\subsection{结果缓存技术}
\subsection{相关性评估技术}
\section{对等网络中的信息搜索}
\subsection{宽松约束的搜索}
\subsection{严格约束的搜索}
\subsection{面向语义的信息搜索}
\section{本章小结}

\chapter{对等网络中宽松约束的一般性搜索的理论模型}
\section{本章引论}
\section{模型基本假设}
\subsection{无偏向性搜索}
\subsection{结点特性}
\subsection{短时稳态性}
\subsection{模型假设的总体叙述}
\section{宽松约束的一般性搜索性能理论模型}
\subsection{单次搜索的带宽开销以及系统总带宽开销}
\subsection{索引分布与搜索开销的关系}
\subsection{索引分布与索引维护开销的关系}
\subsection{搜索总带宽开销和搜索效率的计算公式}
\subsection{模型总体叙述}
\section{模型求解及搜索性能优化}
\subsection{最小化结点的带宽开销}
\subsection{在带宽约束下最优化搜索效率}
\subsection{模型中的参数测定}
\section{模型结论和意义}
\section{相关问题讨论}
\subsection{模型的适用性}
\subsection{与相关工作的比较}
\section{本章小结}

\chapter{近似最优性能的宽松约束搜索算法}
\section{本章引论}
\section{分级组管理(Hierarchical Group Management, HGM)}
\subsection{HGM中的分级量化机制}
\subsection{HGM的体系结构}
\subsection{HGM中的逐级扩展搜索算法}
\subsection{局域性原则(Principle of Locality)}
\subsection{在结构化P2P上构建HGM的方法}
\section{基于Pastry路由基础设施的分级组管理}
\subsection{Pastry上的HGM结点分组结构}
\subsection{基于Pastry路由的逐级扩展和组内局域性消息广播}
\subsection{索引的维护与更新}
\section{基于SkipNet路由基础设施的分级组管理}
\subsection{SkipNet上的HGM结点分组结构}
\subsection{基于SkipNet路由的逐级扩展和组内局域性消息广播}
\subsection{索引的维护与更新}
\section{相关问题讨论}
\subsection{容错问题}
\subsection{索引更新操作的时间}
\section{本章小结}

\chapter{对等网络中严格约束的区域搜索算法}
\section{本章引论}
\section{基于自然属性值匹配的数据存储以及区域搜索}
\section{无中心的资源管理基础设施(DRMI)}
\subsection{结点组的资源元数据信息}
\subsection{资源元数据表(Resource Metadata Table,RMT)}
\subsection{RMT的动态维护}
\section{基于DRMI的资源管理与负载平衡算法}
\subsection{监测任意结点组的资源信息}
\subsection{DRMI上的渐次决策方法}
\subsection{利用DRMI实现负载均衡的自然属性值匹配}
\section{实验结果与分析}
\subsection{资源信息监测的效率和正确性}
\subsection{负载迁移算法}
\section{相关工作}
\section{本章小结}

\chapter{结论}
\section{研究总结}
\section{需进一步开展的工作}

\backmatter
\chapter*{参考文献}
\begin{acknowledgements}
\end{acknowledgements}

\appendix
\chapter{资源元数据表}

\begin{resume}
\end{resume}

\end{document}
