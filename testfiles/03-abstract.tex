\input{regression-test.tex}
\PassOptionsToClass{fontset=fandol}{thuthesis}

\documentclass[degree=doctor]{thuthesis}

\begin{document}
\START
\showoutput

\thusetup{
  ckeywords = {关键词 1, 关键词 2, 关键词 3, 关键词 4, 关键词 5},
}

\begin{cabstract}
  论文的摘要是对论文研究内容和成果的高度概括。摘要应对论文所研究的问题及其研究目的进行描述,对研究方法和过程进行简单介绍,对研究成果和所得结论进行概括。
  摘要应具有独立性和自明性,其内容应包含与论文全文同等量的主要信息。
  使读者即使不阅读全文,通过摘要就能了解论文的总体内容和主要成果。

  论文摘要的书写应力求精确、简明。
  切忌写成对论文书写内容进行提要的形式,尤其要避免“第 1 章……;第 2 章……;……”这种或类似的陈述方式。

  关键词是为了文献标引工作、用以表示全文主要内容信息的单词或术语。
  关键词不超过 5 个,每个关键词中间用分号分隔。
\end{cabstract}

\makeatletter
\thu@makeabstract@zh

\clearpage
\OMIT
\end{document}
