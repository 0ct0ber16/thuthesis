
%%% Local Variables:
%%% mode: latex
%%% TeX-master: t
%%% End:

\chapter{带 English 的标题}
\label{cha:intro}

这是 \thuthesis{} 的示例文档,基本上覆盖了模板中所有格式的设置。建议大家在使用模
板之前,除了阅读《\thuthesis{}用户手册》,这个示例文档也最好能看一看。

小老鼠偷吃热凉粉;短长虫环绕矮高粱。\footnote{韩愈(768-824),字退之,河南河阳(
  今河南孟县)人,自称郡望昌黎,世称韩昌黎。幼孤贫刻苦好学,德宗贞元八年进士。曾
  任监察御史,因上疏请免关中赋役,贬为阳山县令。后随宰相裴度平定淮西迁刑部侍郎,
  又因上表谏迎佛骨,贬潮州刺史。做过吏部侍郎,死谥文公,故世称韩吏部、韩文公。是
  唐代古文运动领袖,与柳宗元合称韩柳。诗力求险怪新奇,雄浑重气势。}


\section{封面相关}
封面的例子请参看 cover.tex。主要符号表参看 denation.tex,附录和个人简历分别参看 appendix01.tex
和 resume.tex。里面的命令都非常简单,一看即会。\footnote{你说还是看不懂?怎么会呢?}

\section{字体命令}
\label{sec:first}

苏轼(1037-1101),北宋文学家、书画家。字子瞻,号东坡居士,眉州眉山(今属四川)人
。苏洵子。嘉佑进士。神宗时曾任祠部员外郎,因反对王安石新法而求外职,任杭州通判,
知密州、徐州、湖州。后以作诗“谤讪朝廷”罪贬黄州。哲宗时任翰林学士,曾出知杭州、
颖州等,官至礼部尚书。后又贬谪惠州、儋州。北还后第二年病死常州。南宋时追谥文忠。
与父洵弟辙,合称“三苏”。在政治上属于旧党,但也有改革弊政的要求。其文汪洋恣肆,
明白畅达,为“唐宋八大家”之一。  其诗清新豪健,善用夸张比喻,在艺术表现方面独具
风格。少数诗篇也能反映民间疾苦,指责统治者的奢侈骄纵。词开豪放一派,对后代很有影
响。《念奴娇·赤壁怀古》、《水调歌头·丙辰中秋》传诵甚广。

{\kaishu 坡仙擅长行书、楷书,取法李邕、徐浩、颜真卿、杨凝式,而能自创新意。用笔丰腴
  跌宕,有天真烂漫之趣。与蔡襄、黄庭坚、米芾并称“宋四家”。能画竹,学文同,也喜
  作枯木怪石。论画主张“神似”,认为“论画以形似,见与儿童邻”;高度评价“诗中有
  画,画中有诗”的艺术造诣。诗文有《东坡七集》等。存世书迹有《答谢民师论文帖》、
  《祭黄几道文》、《前赤壁赋》、《黄州寒食诗帖》等。  画迹有《枯木怪石图》、《
  竹石图》等。}

{\fangsong 易与天地准,故能弥纶天地之道。仰以观於天文,俯以察於地理,是故知幽明之故。原
  始反终,故知死生之说。精气为物,游魂为变,是故知鬼神之情状。与天地相似,故不违。
  知周乎万物,而道济天下,故不过。旁行而不流,乐天知命,故不忧。安土敦乎仁,故
  能爱。范围天地之化而不过,曲成万物而不遗,通乎昼夜之道而知,故神无方而易无体。}

% 非本科生一般用不到幼圆与隶书字体。需要的同学可以使用 ctex-fontset-thuthesis.def 文件
{\ifcsname youyuan\endcsname\youyuan 有天地,然后万物生焉。盈天地之间者,唯万物,
  故受之以屯;屯者盈也,屯者物之始生也。物生必蒙,故受之以蒙;蒙者蒙也,物之穉也。
  物穉不可不养也,故受之以需;需者饮食之道也。饮食必有讼,故受之以讼。讼必有众起,
  故受之以师;师者众也。众必有所比,故受之以比;比者比也。比必有所畜也,故受之以
  小畜。物畜然后有礼,故受之以履。\fi}

{\heiti 履而泰,然后安,故受之以泰;泰者通也。物不可以终通,故受之以否。物不可以终
  否,故受之以同人。与人同者,物必归焉,故受之以大有。有大者不可以盈,故受之以谦。
  有大而能谦,必豫,故受之以豫。豫必有随,故受之以随。以喜随人者,必有事,故受
  之以蛊;蛊者事也。}

{\ifcsname lishu\endcsname\lishu 有事而后可大,故受之以临;临者大也。物大然后可观,
  故受之以观。可观而后有所合,故受之以噬嗑;嗑者合也。物不可以苟合而已,故受之以
  贲;贲者饰也。致饰然后亨,则尽矣,故受之以剥;剥者剥也。物不可以终尽,剥穷上反
  下,故受之以复。复则不妄矣,故受之以无妄。\fi}

{\songti 有无妄然后可畜,故受之以大畜。物畜然后可养,故受之以颐;颐者养也。不养则不
  可动,故受之以大过。物不可以终过,故受之以坎;坎者陷也。陷必有所丽,故受之以
  离;离者丽也。}

\section{表格样本}
\label{chap1:sample:table} 

\subsection{基本表格}
\label{sec:basictable}

模板中关于表格的宏包有三个: \pkg{booktabs}、\pkg{array} 和
\pkg{longtabular},命令有一个 \cs{hlinewd}。三线表可以用 \pkg{booktabs}
提供的 \cs{toprule}、\cs{midrule} 和 \cs{bottomrule}。它们与
\pkg{longtable} 能很好的配合使用。如果表格比较简单的话可以直接用命令
\cs{hlinewd}\marg{width} 控制。
\begin{table}[htb]
  \centering
  \begin{minipage}[t]{0.8\linewidth} % 如果想在表格中使用脚注,minipage是个不错的办法
  \caption[模板文件]{模板文件。如果表格的标题很长,那么在表格索引中就会很不美
    观,所以要像 chapter 那样在前面用中括号写一个简短的标题。这个标题会出现在索
    引中。}
  \label{tab:template-files}
    \begin{tabularx}{\linewidth}{lX}
      \toprule[1.5pt]
      {\heiti 文件名} & {\heiti 描述} \\\midrule[1pt]
      thuthesis.ins & \LaTeX{} 安装文件,docstrip\footnote{表格中的脚注} \\
      thuthesis.dtx & 所有的一切都在这里面\footnote{再来一个}。\\
      thuthesis.cls & 模板类文件。\\
      thuthesis.cfg & 模板配置文。cls 和 cfg 由前两个文件生成。\\
      thuthesis.bst    & 参考文献 Bibtex 样式文件。\\
      thuthesis.sty   & 常用的包和命令写在这里,减轻主文件的负担。\\
      \bottomrule[1.5pt]
    \end{tabularx}
  \end{minipage}
\end{table}

首先来看一个最简单的表格。表 \ref{tab:template-files} 列举了本模板主要文件及其功
能。请大家注意三线表中各条线对应的命令。这个例子还展示了如何在表格中正确使用脚注。
由于 \LaTeX{} 本身不支持在表格中使用 \cs{footnote},所以我们不得不将表格放在
小页中,而且最好将表格的宽度设置为小页的宽度,这样脚注看起来才更美观。

\subsection{复杂表格}
\label{sec:complicatedtable}

我们经常会在表格下方标注数据来源,或者对表格里面的条目进行解释。前面的脚注是一种
不错的方法,如果你不喜欢脚注。那么完全可以在表格后面自己写注释,比如表~\ref{tab:tabexamp1}。
\begin{table}[htbp]
  \centering
  \caption{复杂表格示例 1}
  \label{tab:tabexamp1}
  \begin{minipage}[t]{0.8\textwidth} 
    \begin{tabularx}{\linewidth}{|l|X|X|X|X|}
      \hline
 \multirow{2}*{\diagbox[width=5em]{x}{y}} & \multicolumn{2}{c|}{First Half} & \multicolumn{2}{c|}{Second Half}\\\cline{2-5}
      & 1st Qtr &2nd Qtr&3rd Qtr&4th Qtr \\ \hline
      East$^{*}$ &   20.4&   27.4&   90&     20.4 \\
      West$^{**}$ &   30.6 &   38.6 &   34.6 &  31.6 \\ \hline
    \end{tabularx}\\[2pt]
    \footnotesize 注:数据来源《\thuthesis{} 使用手册》。\\
    *:东部\\
    **:西部
  \end{minipage}
\end{table}

此外,表~\ref{tab:tabexamp1} 同时还演示了另外两个功能:1)通过 \pkg{tabularx} 的
 \texttt{|X|} 扩展实现表格自动放大;2)通过命令 \cs{diagbox} 在表头部分
插入反斜线。

为了使我们的例子更接近实际情况,我会在必要的时候插入一些“无关”文字,以免太多图
表同时出现,导致排版效果不太理想。第一个出场的当然是我的最爱:风流潇洒、骏马绝尘、
健笔凌云的{\heiti 李太白}了。

李白,字太白,陇西成纪人。凉武昭王暠九世孙。或曰山东人,或曰蜀人。白少有逸才,志
气宏放,飘然有超世之心。初隐岷山,益州长史苏颋见而异之,曰:“是子天才英特,可比
相如。”天宝初,至长安,往见贺知章。知章见其文,叹曰:“子谪仙人也。”言于明皇,
召见金銮殿,奏颂一篇。帝赐食,亲为调羹,有诏供奉翰林。白犹与酒徒饮于市,帝坐沉香
亭子,意有所感,欲得白为乐章,召入,而白已醉。左右以水颒面,稍解,援笔成文,婉丽
精切。帝爱其才,数宴见。白常侍帝,醉,使高力士脱靴。力士素贵,耻之,摘其诗以激杨
贵妃。帝欲官白,妃辄沮止。白自知不为亲近所容,恳求还山。帝赐金放还。乃浪迹江湖,
终日沉饮。永王璘都督江陵,辟为僚佐。璘谋乱,兵败,白坐长流夜郎,会赦得还。族人阳
冰为当涂令,白往依之。代宗立,以左拾遗召,而白已卒。文宗时,诏以白歌诗、裴旻剑舞、
张旭草书为三绝云。集三十卷。今编诗二十五卷。\hfill —— 《全唐诗》诗人小传

浮动体的并排放置一般有两种情况:1)二者没有关系,为两个独立的浮动体;2)二者隶属
于同一个浮动体。对表格来说并排表格既可以像图~\ref{tab:parallel1}、图~\ref{tab:parallel2} 
使用小页环境,也可以如图~\ref{tab:subtable} 使用子表格来做。图的例子参见第~\ref{sec:multifig} 节。

\begin{table}[htbp]
\noindent\begin{minipage}{0.5\textwidth}
\centering
\caption{第一个并排子表格}
\label{tab:parallel1}
\begin{tabular}{p{2cm}p{2cm}}
\toprule[1.5pt]
111 & 222 \\\midrule[1pt]
222 & 333 \\\bottomrule[1.5pt]
\end{tabular}
\end{minipage}%
\begin{minipage}{0.5\textwidth}
\centering
\caption{第二个并排子表格}
\label{tab:parallel2}
\begin{tabular}{p{2cm}p{2cm}}
\toprule[1.5pt]
111 & 222 \\\midrule[1pt]
222 & 333 \\\bottomrule[1.5pt]
\end{tabular}
\end{minipage}
\end{table}

然后就是忧国忧民,诗家楷模杜工部了。杜甫,字子美,其先襄阳人,曾祖依艺为巩令,因
居巩。甫天宝初应进士,不第。后献《三大礼赋》,明皇奇之,召试文章,授京兆府兵曹参
军。安禄山陷京师,肃宗即位灵武,甫自贼中遁赴行在,拜左拾遗。以论救房琯,出为华州
司功参军。关辅饥乱,寓居同州同谷县,身自负薪采梠,餔糒不给。久之,召补京兆府功曹,
道阻不赴。严武镇成都,奏为参谋、检校工部员外郎,赐绯。武与甫世旧,待遇甚厚。乃于
成都浣花里种竹植树,枕江结庐,纵酒啸歌其中。武卒,甫无所依,乃之东蜀就高適。既至
而適卒。是岁,蜀帅相攻杀,蜀大扰。甫携家避乱荆楚,扁舟下峡,未维舟而江陵亦乱。乃
溯沿湘流,游衡山,寓居耒阳。卒年五十九。元和中,归葬偃师首阳山,元稹志其墓。天宝
间,甫与李白齐名,时称李杜。然元稹之言曰:“李白壮浪纵恣,摆去拘束,诚亦差肩子美
矣。至若铺陈终始,排比声韵,大或千言,次犹数百,词气豪迈,而风调清深,属对律切,
而脱弃凡近,则李尚不能历其藩翰,况堂奥乎。”白居易亦云:“杜诗贯穿古今,  尽工尽
善,殆过于李。”元、白之论如此。盖其出处劳佚,喜乐悲愤,好贤恶恶,一见之于诗。而
又以忠君忧国、伤时念乱为本旨。读其诗可以知其世,故当时谓之“诗史”。旧集诗文共六
十卷,今编诗十九卷。

\begin{table}[htbp]
\centering
\caption{并排子表格}
\label{tab:subtable}
\subcaptionbox{第一个子表格}
{
\begin{tabular}{p{2cm}p{2cm}}
\toprule[1.5pt]
111 & 222 \\\midrule[1pt]
222 & 333 \\\bottomrule[1.5pt]
\end{tabular}
}
\hskip2cm
\subcaptionbox{第二个子表格}
{
\begin{tabular}{p{2cm}p{2cm}}
\toprule[1.5pt]
111 & 222 \\\midrule[1pt]
222 & 333 \\\bottomrule[1.5pt]
\end{tabular}
}
\end{table}

不可否认 \LaTeX{} 的表格功能没有想象中的那么强大,不过只要你足够认真,足够细致,那么
同样可以排出来非常复杂非常漂亮的表格。请参看表~\ref{tab:tabexamp2}。
\begin{table}[htbp]
  \centering\dawu[1.3]
  \caption{复杂表格示例 2}
  \label{tab:tabexamp2}
  \begin{tabular}[c]{|m{1.5cm}|c|c|c|c|c|c|}\hline
    \multicolumn{2}{|c|}{Network Topology} & \# of nodes & 
    \multicolumn{3}{c|}{\# of clients} & Server \\\hline
    GT-ITM & Waxman Transit-Stub & 600 &
    \multirow{2}{2em}{2\%}& 
    \multirow{2}{2em}{10\%}& 
    \multirow{2}{2em}{50\%}& 
    \multirow{2}{1.2in}{Max. Connectivity}\\\cline{1-3}
    \multicolumn{2}{|c|}{Inet-2.1} & 6000 & & & &\\\hline
    \multirow{2}{1.5cm}{Xue} & Rui  & Ni &\multicolumn{4}{c|}{\multirow{2}*{\thuthesis}}\\\cline{2-3}
    & \multicolumn{2}{c|}{ABCDEF} &\multicolumn{4}{c|}{} \\\hline
\end{tabular}
\end{table}

最后就是清新飘逸、文约意赅、空谷绝响的王大侠了。王维,字摩诘,河东人。工书画,与
弟缙俱有俊才。开元九年,进士擢第,调太乐丞。坐累为济州司仓参军,历右拾遗、监察御
史、左补阙、库部郎中,拜吏部郎中。天宝末,为给事中。安禄山陷两都,维为贼所得,服
药阳喑,拘于菩提寺。禄山宴凝碧池,维潜赋诗悲悼,闻于行在。贼平,陷贼官三等定罪,
特原之,责授太子中允,迁中庶子、中书舍人。复拜给事中,转尚书右丞。维以诗名盛于开
元、天宝间,宁薛诸王驸马豪贵之门,无不拂席迎之。得宋之问辋川别墅,山水绝胜,与道
友裴迪,浮舟往来,弹琴赋诗,啸咏终日。笃于奉佛,晚年长斋禅诵。一日,忽索笔作书
数纸,别弟缙及平生亲故,舍笔而卒。赠秘书监。宝应中,代宗问缙:“朕常于诸王坐闻维
乐章,今存几何?”缙集诗六卷,文四卷,表上之。敕答云,卿伯氏位列先朝,名高希代。
抗行周雅,长揖楚辞。诗家者流,时论归美。克成编录,叹息良深。殷璠谓维诗词秀调雅,
意新理惬。在泉成珠,著壁成绘。苏轼亦云:“维诗中有画,画中有诗也。”今编诗四卷。

要想用好论文模板还是得提前学习一些 \TeX/\LaTeX{}的相关知识,具备一些基本能力,掌
握一些常见技巧,否则一旦遇到问题还真是比较麻烦。我们见过很多这样的同学,一直以来
都是使用 Word 等字处理工具,以为 \LaTeX{}模板的用法也应该类似,所以就沿袭同样的思
路来对待这种所见非所得的排版工具,结果被折腾的焦头烂额,疲惫不堪。

如果您要排版的表格长度超过一页,那么推荐使用 \pkg{longtable} 或者 \pkg{supertabular}
宏包,模板对 \pkg{longtable} 进行了相应的设置,所以用起来可能简单一些。
表~\ref{tab:performance} 就是 \pkg{longtable} 的简单示例。
\begin{longtable}[c]{c*{6}{r}}
\caption{实验数据}\label{tab:performance}\\
\toprule[1.5pt]
 测试程序 & \multicolumn{1}{c}{正常运行} & \multicolumn{1}{c}{同步} & \multicolumn{1}{c}{检查点} & \multicolumn{1}{c}{卷回恢复}
& \multicolumn{1}{c}{进程迁移} & \multicolumn{1}{c}{检查点} \\
& \multicolumn{1}{c}{时间 (s)}& \multicolumn{1}{c}{时间 (s)}&
\multicolumn{1}{c}{时间 (s)}& \multicolumn{1}{c}{时间 (s)}& \multicolumn{1}{c}{
  时间 (s)}&  文件(KB)\\\midrule[1pt]
\endfirsthead
\multicolumn{7}{c}{续表~\thetable\hskip1em 实验数据}\\
\toprule[1.5pt]
 测试程序 & \multicolumn{1}{c}{正常运行} & \multicolumn{1}{c}{同步} & \multicolumn{1}{c}{检查点} & \multicolumn{1}{c}{卷回恢复}
& \multicolumn{1}{c}{进程迁移} & \multicolumn{1}{c}{检查点} \\
& \multicolumn{1}{c}{时间 (s)}& \multicolumn{1}{c}{时间 (s)}&
\multicolumn{1}{c}{时间 (s)}& \multicolumn{1}{c}{时间 (s)}& \multicolumn{1}{c}{
  时间 (s)}&  文件(KB)\\\midrule[1pt]
\endhead
\hline
\multicolumn{7}{r}{续下页}
\endfoot
\endlastfoot
CG.A.2 & 23.05 & 0.002 & 0.116 & 0.035 & 0.589 & 32491 \\
CG.A.4 & 15.06 & 0.003 & 0.067 & 0.021 & 0.351 & 18211 \\
CG.A.8 & 13.38 & 0.004 & 0.072 & 0.023 & 0.210 & 9890 \\
CG.B.2 & 867.45 & 0.002 & 0.864 & 0.232 & 3.256 & 228562 \\
CG.B.4 & 501.61 & 0.003 & 0.438 & 0.136 & 2.075 & 123862 \\
CG.B.8 & 384.65 & 0.004 & 0.457 & 0.108 & 1.235 & 63777 \\
MG.A.2 & 112.27 & 0.002 & 0.846 & 0.237 & 3.930 & 236473 \\
MG.A.4 & 59.84 & 0.003 & 0.442 & 0.128 & 2.070 & 123875 \\
MG.A.8 & 31.38 & 0.003 & 0.476 & 0.114 & 1.041 & 60627 \\
MG.B.2 & 526.28 & 0.002 & 0.821 & 0.238 & 4.176 & 236635 \\
MG.B.4 & 280.11 & 0.003 & 0.432 & 0.130 & 1.706 & 123793 \\
MG.B.8 & 148.29 & 0.003 & 0.442 & 0.116 & 0.893 & 60600 \\
LU.A.2 & 2116.54 & 0.002 & 0.110 & 0.030 & 0.532 & 28754 \\
LU.A.4 & 1102.50 & 0.002 & 0.069 & 0.017 & 0.255 & 14915 \\
LU.A.8 & 574.47 & 0.003 & 0.067 & 0.016 & 0.192 & 8655 \\
LU.B.2 & 9712.87 & 0.002 & 0.357 & 0.104 & 1.734 & 101975 \\
LU.B.4 & 4757.80 & 0.003 & 0.190 & 0.056 & 0.808 & 53522 \\
LU.B.8 & 2444.05 & 0.004 & 0.222 & 0.057 & 0.548 & 30134 \\
EP.A.2 & 123.81 & 0.002 & 0.010 & 0.003 & 0.074 & 1834 \\
EP.A.4 & 61.92 & 0.003 & 0.011 & 0.004 & 0.073 & 1743 \\
EP.A.8 & 31.06 & 0.004 & 0.017 & 0.005 & 0.073 & 1661 \\
EP.B.2 & 495.49 & 0.001 & 0.009 & 0.003 & 0.196 & 2011 \\
EP.B.4 & 247.69 & 0.002 & 0.012 & 0.004 & 0.122 & 1663 \\
EP.B.8 & 126.74 & 0.003 & 0.017 & 0.005 & 0.083 & 1656 \\
\bottomrule[1.5pt]
\end{longtable}

\subsection{其它}
\label{sec:tableother}
如果不想让某个表格或者图片出现在索引里面,请使用命令 \cs{caption*}。
这个命令不会给表格编号,也就是出来的只有标题文字而没有“表~XX”,“图~XX”,否则
索引里面序号不连续就显得不伦不类,这也是 \LaTeX{} 里星号命令默认的规则。

有这种需求的多是本科同学的英文资料翻译部分,如果你觉得附录中英文原文中的表格和图
片显示成“  表”和“图”很不协调的话,一个很好的办法就是用 \cs{caption*},参数
随便自己写,比如不守规矩的表~1.111 和图~1.111 能满足这种特殊需要(可以参看附录部
分)。
\begin{table}[ht]
  \begin{minipage}{0.4\linewidth}
    \centering
    \caption*{表~1.111\quad 这是一个手动编号,不出现在索引中的表格。}
    \label{tab:badtabular}
      \framebox(150,50)[c]{\thuthesis}
  \end{minipage}%
  \hfill%
  \begin{minipage}{0.4\linewidth}
    \centering
    \caption*{Figure~1.111\quad 这是一个手动编号,不出现在索引中的图。}
    \label{tab:badfigure}
    \framebox(150,50)[c]{薛瑞尼}
  \end{minipage}
\end{table}

如果你的确想让它编号,但又不想让它出现在索引中的话,那就自己看看代码改一改吧,我
目前不打算给模板增加这种另类命令。

最后,虽然大家不一定会独立使用小页,但是关于小页中的脚注还是有必要提一下。请看下
面的例子。

\begin{minipage}[t]{\linewidth-2\parindent}
  柳宗元,字子厚(773-819),河东(今永济县)人\footnote{山西永济水饺。},是唐代
  杰出的文学家,哲学家,同时也是一位政治改革家。与韩愈共同倡导唐代古文运动,并称
  韩柳\footnote{唐宋八大家之首二位。}。
\end{minipage}

唐朝安史之乱后,宦官专权,藩镇割据,土地兼并日渐严重,社会生产破坏严重,民不聊生。柳宗
元对这种社会现实极为不满,他积极参加了王叔文领导的“永济革新”,并成为这一
运动的中坚人物。他们革除弊政,打击权奸,触犯了宦官和官僚贵族利益,在他们的联合反
扑下,改革失败了,柳宗元被贬为永州司马。

\section{定理环境}
\label{sec:theorem}

给大家演示一下各种和证明有关的环境:

\begin{assumption}
待月西厢下,迎风户半开;隔墙花影动,疑是玉人来。
\begin{eqnarray}
  \label{eq:eqnxmp}
  c & = & a^2 - b^2\\
    & = & (a+b)(a-b)
\end{eqnarray}
\end{assumption}

千辛万苦,历尽艰难,得有今日。然相从数千里,未曾哀戚。今将渡江,方图百年欢笑,如
何反起悲伤?(引自《杜十娘怒沉百宝箱》)

\begin{definition}
子曰:「道千乘之国,敬事而信,节用而爱人,使民以时。」
\end{definition}

千古第一定义!问世间、情为何物,只教生死相许?天南地北双飞客,老翅几回寒暑。欢乐趣,离别苦,就中更有痴儿女。
君应有语,渺万里层云,千山暮雪,只影向谁去?

横汾路,寂寞当年箫鼓,荒烟依旧平楚。招魂楚些何嗟及,山鬼暗谛风雨。天也妒,未信与,莺儿燕子俱黄土。
千秋万古,为留待骚人,狂歌痛饮,来访雁丘处。

\begin{proposition}
 曾子曰:「吾日三省吾身 —— 为人谋而不忠乎?与朋友交而不信乎?传不习乎?」
\end{proposition}

多么凄美的命题啊!其日牛马嘶,新妇入青庐,奄奄黄昏后,寂寂人定初,我命绝今日,
魂去尸长留,揽裙脱丝履,举身赴清池,府吏闻此事,心知长别离,徘徊庭树下,自挂东南
枝。

\begin{remark}
天不言自高,水不言自流。
\begin{gather*}
\begin{split} 
\varphi(x,z)
&=z-\gamma_{10}x-\gamma_{mn}x^mz^n\\
&=z-Mr^{-1}x-Mr^{-(m+n)}x^mz^n
\end{split}\\[6pt]
\begin{align} \zeta^0&=(\xi^0)^2,\\
\zeta^1 &=\xi^0\xi^1,\\
\zeta^2 &=(\xi^1)^2,
\end{align}
\end{gather*}
\end{remark}

天尊地卑,乾坤定矣。卑高以陈,贵贱位矣。 动静有常,刚柔断矣。方以类聚,物以群分,
吉凶生矣。在天成象,在地成形,变化见矣。鼓之以雷霆,润之以风雨,日月运行,一寒一
暑,乾道成男,坤道成女。乾知大始,坤作成物。乾以易知,坤以简能。易则易知,简则易
从。易知则有亲,易从则有功。有亲则可久,有功则可大。可久则贤人之德,可大则贤人之
业。易简,而天下矣之理矣;天下之理得,而成位乎其中矣。

\begin{axiom}
两点间直线段距离最短。  
\begin{align}
x&\equiv y+1\pmod{m^2}\\
x&\equiv y+1\mod{m^2}\\
x&\equiv y+1\pod{m^2}
\end{align}
\end{axiom}

《彖曰》:大哉乾元,万物资始,乃统天。云行雨施,品物流形。大明始终,六位时成,时
乘六龙以御天。乾道变化,各正性命,保合大和,乃利贞。首出庶物,万国咸宁。

《象曰》:天行健,君子以自强不息。潜龙勿用,阳在下也。见龙再田,德施普也。终日乾
乾,反复道也。或跃在渊,进无咎也。飞龙在天,大人造也。亢龙有悔,盈不可久也。用九,
天德不可为首也。   

\begin{lemma}
《猫和老鼠》是我最爱看的动画片。
\begin{multline*}%\tag*{[a]} % 这个不出现在索引中
\int_a^b\biggl\{\int_a^b[f(x)^2g(y)^2+f(y)^2g(x)^2]
 -2f(x)g(x)f(y)g(y)\,dx\biggr\}\,dy \\
 =\int_a^b\biggl\{g(y)^2\int_a^bf^2+f(y)^2
  \int_a^b g^2-2f(y)g(y)\int_a^b fg\biggr\}\,dy
\end{multline*}
\end{lemma}

行行重行行,与君生别离。相去万余里,各在天一涯。道路阻且长,会面安可知。胡马依北
风,越鸟巢南枝。相去日已远,衣带日已缓。浮云蔽白日,游子不顾返。思君令人老,岁月
忽已晚。  弃捐勿复道,努力加餐饭。

\begin{theorem}\label{the:theorem1}
犯我强汉者,虽远必诛\hfill —— 陈汤(汉)
\end{theorem}
\begin{subequations}
\begin{align}
y & = 1 \\
y & = 0
\end{align}
\end{subequations}
道可道,非常道。名可名,非常名。无名天地之始;有名万物之母。故常无,欲以观其妙;
常有,欲以观其徼。此两者,同出而异名,同谓之玄。玄之又玄,众妙之门。上善若水。水
善利万物而不争,处众人之所恶,故几于道。曲则全,枉则直,洼则盈,敝则新,少则多,
多则惑。人法地,地法天,天法道,道法自然。知人者智,自知者明。胜人者有力,自胜
者强。知足者富。强行者有志。不失其所者久。死而不亡者寿。

\begin{proof}
燕赵古称多感慨悲歌之士。董生举进士,连不得志于有司,怀抱利器,郁郁适兹土,吾
知其必有合也。董生勉乎哉?

夫以子之不遇时,苟慕义强仁者,皆爱惜焉,矧燕、赵之士出乎其性者哉!然吾尝闻
风俗与化移易,吾恶知其今不异于古所云邪?聊以吾子之行卜之也。董生勉乎哉?

吾因子有所感矣。为我吊望诸君之墓,而观于其市,复有昔时屠狗者乎?为我谢
曰:“明天子在上,可以出而仕矣!” \hfill —— 韩愈《送董邵南序》
\end{proof}

\begin{corollary}
  四川话配音的《猫和老鼠》是世界上最好看最好听最有趣的动画片。
\begin{alignat}{3}
V_i & =v_i - q_i v_j, & \qquad X_i & = x_i - q_i x_j,
 & \qquad U_i & = u_i,
 \qquad \text{for $i\ne j$;}\label{eq:B}\\
V_j & = v_j, & \qquad X_j & = x_j,
  & \qquad U_j & u_j + \sum_{i\ne j} q_i u_i.
\end{alignat}
\end{corollary}

迢迢牵牛星,皎皎河汉女。
纤纤擢素手,札札弄机杼。
终日不成章,泣涕零如雨。
河汉清且浅,相去复几许。
盈盈一水间,脉脉不得语。

\begin{example}
  大家来看这个例子。
\begin{equation}
\label{ktc}
\left\{\begin{array}{l}
\nabla f({\mbox{\boldmath $x$}}^*)-\sum\limits_{j=1}^p\lambda_j\nabla g_j({\mbox{\boldmath $x$}}^*)=0\\[0.3cm]
\lambda_jg_j({\mbox{\boldmath $x$}}^*)=0,\quad j=1,2,\cdots,p\\[0.2cm]
\lambda_j\ge 0,\quad j=1,2,\cdots,p.
\end{array}\right.
\end{equation}
\end{example}

\begin{exercise}
  清列出 Andrew S. Tanenbaum 和 W. Richard Stevens 的所有著作。
\end{exercise}

\begin{conjecture} \textit{Poincare Conjecture} If in a closed three-dimensional
  space, any closed curves can shrink to a point continuously, this space can be
  deformed to a sphere.
\end{conjecture}

\begin{problem}
 回答还是不回答,是个问题。 
\end{problem}

如何引用定理~\ref{the:theorem1} 呢?加上 \cs{label} 使用 \cs{ref} 即可。妾发
初覆额,折花门前剧。郎骑竹马来,绕床弄青梅。同居长干里,两小无嫌猜。 十四为君妇,
羞颜未尝开。低头向暗壁,千唤不一回。十五始展眉,愿同尘与灰。常存抱柱信,岂上望夫
台。 十六君远行,瞿塘滟滪堆。五月不可触,猿声天上哀。门前迟行迹,一一生绿苔。苔深
不能扫,落叶秋风早。八月蝴蝶来,双飞西园草。感此伤妾心,坐愁红颜老。

\section{参考文献}
\label{sec:bib}
当然参考文献可以直接写 bibitem,虽然费点功夫,但是好控制,各种格式可以自己随意改
写。

本模板推荐使用 BIB\TeX,样式文件为 \texttt{thuthesis.bst},基本符合学校的参考文献格
式(如专利等引用未加详细测试)。看看这个例子,关于书的~\cite{tex, companion,
  ColdSources},还有这些~\cite{Krasnogor2004e, clzs, zjsw},关于杂志
的~\cite{ELIDRISSI94, MELLINGER96, SHELL02},硕士论文~\cite{zhubajie,
  metamori2004},博士论文~\cite{shaheshang, FistSystem01},标准文
件~\cite{IEEE-1363},会议论文~\cite{DPMG,kocher99},技术报告~\cite{NPB2},电子文
献~\cite{chuban2001,oclc2000}。中文参考文献~\cite{cnarticle}应增
加 \texttt{lang=``zh''} 字段,以便进行相应处理。另外,本模板对中文文
献~\cite{cnproceed}的支持并不是十全十美,如果有不如意的地方,请手动修
改 \texttt{bbl} 文件。

有时候不想要上标,那么可以这样~\inlinecite{shaheshang},这个非常重要。

有时候一些参考文献没有纸质出处,需要标注 URL。缺省情况下,URL 不会在连字符处断行,
这可能使得用连字符代替空格的网址分行很难看。如果需要,可以将模板类文件中
\begin{verbatim}
\RequirePackage{hyperref}
\end{verbatim}
一行改为:
\begin{verbatim}
\PassOptionsToPackage{hyphens}{url}
\RequirePackage{hyperref}
\end{verbatim}
使得连字符处可以断行。更多设置可以参考 \texttt{url} 宏包文档。

\section{公式}
\label{sec:equation}
贝叶斯公式如式~(\ref{equ:chap1:bayes}),其中 $p(y|\mathbf{x})$ 为后验;
$p(\mathbf{x})$ 为先验;分母 $p(\mathbf{x})$ 为归一化因子。
\begin{equation}
\label{equ:chap1:bayes}
p(y|\mathbf{x}) = \frac{p(\mathbf{x},y)}{p(\mathbf{x})}=
\frac{p(\mathbf{x}|y)p(y)}{p(\mathbf{x})} 
\end{equation}

论文里面公式越多,\TeX{} 就越 happy。再看一个 \pkg{amsmath} 的例子:
\newcommand{\envert}[1]{\left\lvert#1\right\rvert} 
\begin{equation}\label{detK2}
\det\mathbf{K}(t=1,t_1,\dots,t_n)=\sum_{I\in\mathbf{n}}(-1)^{\envert{I}}
\prod_{i\in I}t_i\prod_{j\in I}(D_j+\lambda_jt_j)\det\mathbf{A}
^{(\lambda)}(\overline{I}|\overline{I})=0.
\end{equation} 

前面定理示例部分列举了很多公式环境,可以说把常见的情况都覆盖了,大家在写公式的时
候一定要好好看 \pkg{amsmath} 的文档,并参考模板中的用法:
\begin{multline*}%\tag{[b]} % 这个出现在索引中的
\int_a^b\biggl\{\int_a^b[f(x)^2g(y)^2+f(y)^2g(x)^2]
 -2f(x)g(x)f(y)g(y)\,dx\biggr\}\,dy \\
 =\int_a^b\biggl\{g(y)^2\int_a^bf^2+f(y)^2
  \int_a^b g^2-2f(y)g(y)\int_a^b fg\biggr\}\,dy
\end{multline*}

其实还可以看看这个多级规划:
\begin{equation}\label{bilevel}
\left\{\begin{array}{l}
\max\limits_{{\mbox{\footnotesize\boldmath $x$}}} F(x,y_1^*,y_2^*,\cdots,y_m^*)\\[0.2cm]
\mbox{subject to:}\\[0.1cm]
\qquad G(x)\le 0\\[0.1cm]
\qquad(y_1^*,y_2^*,\cdots,y_m^*)\mbox{ solves problems }(i=1,2,\cdots,m)\\[0.1cm]
\qquad\left\{\begin{array}{l}
    \max\limits_{{\mbox{\footnotesize\boldmath $y_i$}}}f_i(x,y_1,y_2,\cdots,y_m)\\[0.2cm]
    \mbox{subject to:}\\[0.1cm]
    \qquad g_i(x,y_1,y_2,\cdots,y_m)\le 0.
    \end{array}\right.
\end{array}\right.
\end{equation}
这些跟规划相关的公式都来自于刘宝碇老师《不确定规划》的课件。
