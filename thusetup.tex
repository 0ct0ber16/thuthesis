% !TeX root = ./thuthesis-example.tex

% 论文基本信息配置

\thusetup{
  %******************************
  % 注意:
  %   1. 配置里面不要出现空行
  %   2. 不需要的配置信息可以删除
  %   3. 建议先阅读文档中所有关于选项的说明
  %******************************
  %
  % 输出格式
  %   选择打印版(print)或用于提交的电子版(electronic),前者会插入空白页以便直接双面打印
  %
  output = print,
  %
  % 标题
  %   可使用“\\”命令手动控制换行
  %
  title  = {清华大学学位论文 \LaTeX{} 模板\\使用示例文档 v\version},
  title* = {An Introduction to \LaTeX{} Thesis Template of Tsinghua
            University v\version},
  %
  % 学位
  %   1. 学术型
  %      - 中文
  %        需注明所属的学科门类,例如:
  %        哲学、经济学、法学、教育学、文学、历史学、理学、工学、农学、医学、
  %        军事学、管理学、艺术学
  %      - 英文
  %        博士:Doctor of Philosophy
  %        硕士:
  %          哲学、文学、历史学、法学、教育学、艺术学门类,公共管理学科
  %          填写“Master of Arts“,其它填写“Master of Science”
  %   2. 专业型
  %      直接填写专业学位的名称,例如:
  %      教育博士、工程硕士等
  %      Doctor of Education, Master of Engineering
  %   3. 本科生不需要填写
  %
  degree-name  = {工学硕士},
  degree-name* = {Master of Science},
  %
  % 培养单位
  %   填写所属院系的全名
  %
  department = {计算机科学与技术系},
  %
  % 学科
  %   1. 学术型学位
  %      获得一级学科授权的学科填写一级学科名称,其他填写二级学科名称
  %   2. 工程硕士
  %      工程领域名称
  %   3. 其他专业型学位
  %      不填写此项
  %   4. 本科生填写专业名称,第二学位论文需标注“(第二学位)”
  %
  discipline  = {计算机科学与技术},
  discipline* = {Computer Science and Technology},
  %
  % 姓名
  %
  author  = {薛瑞尼},
  author* = {Xue Ruini},
  %
  % 指导教师
  %   中文姓名和职称之间以英文逗号“,”分开,下同
  %
  supervisor  = {郑纬民, 教授},
  supervisor* = {Professor Zheng Weimin},
  %
  % 副指导教师
  %
  associate-supervisor  = {陈文光, 教授},
  associate-supervisor* = {Professor Chen Wenguang},
  %
  % 联合指导教师
  %
  % joint-supervisor  = {某某某, 教授},
  % joint-supervisor* = {Professor Mou Moumou},
  %
  % 日期
  %   使用 ISO 格式;默认为当前时间
  %
  % date = {2019-07-07},
  %
  % 是否在中文封面后的空白页生成书脊(默认 false)
  %
  include-spine = false,
  %
  % 生成的声明页是否要插入页眉和页脚(默认 empty)
  % 仅在需要进行电子签名时,才需要打开这一选项
  % 插入的扫描声明页总是会生成页眉(研究生)和页脚,不受这一选项影响
  %
  % statement-page-style = plain,
  %
  % 密级和年限
  %   秘密, 机密, 绝密
  %
  % secret-level = {秘密},
  % secret-year  = {10},
  %
  % 博士后专有部分
  %
  % clc                = {分类号},
  % udc                = {UDC},
  % id                 = {编号},
  % discipline-level-1 = {计算机科学与技术},  % 流动站(一级学科)名称
  % discipline-level-2 = {系统结构},          % 专业(二级学科)名称
  % start-date         = {2011-07-01},        % 研究工作起始时间
}

%% Put any packages you would like to use here

% 表格中支持跨行
\usepackage{multirow}

% 跨页表格
\usepackage{longtable}

% 固定宽度的表格。放在 hyperref 之前的话,tabularx 里的 footnote 显示不出来。
\usepackage{tabularx}

% 表格加脚注
\usepackage{threeparttable}
\pretocmd{\TPTnoteSettings}{\footnotesize}{}{}

% 确定浮动对象的位置,可以使用 H,强制将浮动对象放到这里(可能效果很差)
\usepackage{float}

% 浮动图形控制宏包。
% 允许上一个 section 的浮动图形出现在下一个 section 的开始部分
% 该宏包提供处理浮动对象的 \FloatBarrier 命令,使所有未处
% 理的浮动图形立即被处理。这三个宏包仅供参考,未必使用:
% \usepackage[below]{placeins}
% \usepackage{floatflt} % 图文混排用宏包
% \usepackage{rotating} % 图形和表格的控制旋转

% 定理类环境宏包
\usepackage{amsthm}
% 也可以使用 ntheorem
% \usepackage[amsmath,thmmarks,hyperref]{ntheorem}

% 给自定义的宏后面自动加空白
% \usepackage{xspace}

% 定义所有的图片文件在 figures 子目录下
\graphicspath{{figures/}}

% 定义自己常用的东西
% \def\myname{薛瑞尼}

% 数学命令
\newcommand\dif{\mathop{}\!\mathrm{d}}  % 微分符号
\newcommand\real{{\mathbf{R}}}  % 实数集
\newcommand\abs[1]{\lvert#1\rvert}
\newcommand\VECTOR{\symbf}  % 向量
\newcommand\MATRIX{\symbf}  % 矩阵
\newcommand\vn{{\VECTOR{n}}}
\newcommand\vx{{\VECTOR{x}}}
\newcommand\mA{{\MATRIX{A}}}
\newcommand\mK{{\MATRIX{K}}}

% 借用 ltxdoc 里面的几个命令方便写文档。
\DeclareRobustCommand\cs[1]{\texttt{\char`\\#1}}
\providecommand\pkg[1]{{\sffamily#1}}

% hyperref 宏包在最后调用
\usepackage{hyperref}
